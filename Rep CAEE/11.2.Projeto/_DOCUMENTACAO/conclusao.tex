Sobre o circuito nota-se que seu funcionamento � total, por�m h� muita perda de precis�o pelo fato de ter-se poucos LEDs para exibi��o do resultado. Com dezesseis LEDs para a mantissa e talvez somente mais um para o expoente haveria uma maior precis�o.

Relativo ao conhecimento � certo que ap�s todo o desenvolvimento do projeto v�rias d�vidas relativas a VHDL foram sanadas. Aprendeu-se uma das v�rias aplica��es do VHDL na ind�stria, embora � de conhecimento da equipe que n�o � bem assim que os sistemas industriais s�o efetuados pois esse projeto, por exemplo, n�o foi bem otimizado, por�m j� � poss�vel ter-se uma id�ia de como projetos desse tipo s�o elaborados.

Al�m da pr�tica, a parte te�rica da disciplina foi bem aprendida pois com o projeto foi poss�vel aprender melhor o funcionamento de registradores, m�quinas de estados, circuitos sequenciais e outros. 